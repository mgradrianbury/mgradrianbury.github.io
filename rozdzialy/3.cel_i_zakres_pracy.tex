\chapter{Cel i zakres pracy}

Rozpoznawanie twarzy to jedna z dziedzin informatyki, która polega na zdolności
do rozpoznawania i identyfikacji tożsamości osoby na podstawie określonych wzorców~\cite{william2019face}.
Innymi słowy, rozpoznawanie twarzy to wykrycie obszaru na fotografii,
w którym znajduje się twarz oraz weryfikacja tożsamości.

Problem identyfikacji tożsamości jest ogólnie znany jako problem klasyfikacji,
czyli określenie etykiety dostarczonego zdjęcia twarzy na podstawie aktualnego zbioru zdjęć.
Problem może być postrzegany jako zapytanie systemu ``kim jest osoba przedstawiona na zdjęciu?''.
Przykładem procesu, gdzie jest stosowana identyfikacja, to proces wyszukiwana tożsamości.
Po dostarczeniu pliku system przeszukuje bazę dostępnych zdjęć w celu znalezienia pasującego zdjęcia
i jeżeli zdjęcie zostanie dopasowane, to system zwróci informację o tożsamości.

Weryfikacja to proces sprawdzający prawdziwość, przydatność lub prawidłowość czegoś~\cite{sjp_pwn_1996}.
W przypadku weryfikacji tożsamości na podstawie zdjęcia twarzy, proces ten będzie polegał na sprawdzeniu,
czy deklarowana tożsamość zgadza się z przypisanym do niej zdjęciem twarzy, jeżeli tak,
to akceptowanie, jeżeli nie to odrzucanie oświadczenia.
Zapytanie, jakie zadaje się wtedy systemowi to ``czy to zdjęcie przedstawia tę osobę?''.
Przykładem weryfikacji tożsamości za pomocą twarzy jest odblokowywanie smartfonów z wykorzystaniem kamery.
Aplikacja w smartfonie sprawdza wtedy, czy zdjęcie osoby z kamery zgadza się ze zdjęciem zapisanym w pamięci.
Jeżeli zdjęcia, zapisane w pamięci oraz dostarczone z kamery, zgadzają się, to telefon zostanie odblokowany.

System, który implementuje weryfikację lub identyfikację na podstawie twarzy,
może znaleźć zastosowanie do wykonywania automatycznego oznaczania osób na zdjęciach~\cite{facebook-aut-tag},
przeszukiwania stron internetowych w celu znalezienie zdjęcia przedstawiającego daną osobę~\cite{pimeyes}
czy zautomatyzowanego grupowania zdjęć na podstawie twarzy na ich się znajdujących~\cite{google-photos_groupby}.

Celem pracy dyplomowej jest utworzenie systemu, którego zadaniem będzie rozpoznanie osoby przedstawionej na zdjęciu.
Zakres prac obejmuje implementację strony internetowej,
za pomocą której użytkownik będzie miał możliwość przesłania wybranego przez siebie zdjęcia,
oraz systemu, który zwróci informację na temat osoby widniejącej na owym zdjęciu
(przesyłanym za pomocą strony internetowej),
pod warunkiem, że ta osoba zostanie znaleziona w bazie zdjęć.
W systemie zostanie użyta głęboka sieć neuronowa o architekturze FaceNet do badania
podobieństwa pomiędzy poszczególnymi twarzami.
Aby spełnić wszystkie oczekiwania, które zostały postawione systemowi, należy wykonać następujące kroki:

\begin{enumerate}
    \item sformułować koncepcję działania systemu,
    \item wybrać odpowiednie technologie oraz narzędzia,
    \item zaimplementować poszczególne elementy systemu,
    \item przetestować system
\end{enumerate}
