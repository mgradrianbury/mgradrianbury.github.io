\chapter*{Podsumowanie}
\addcontentsline{toc}{chapter}{Podsumowanie}

Użytkownik, dzięki temu, że komunikacja z aplikacją odbywa się przez protokół HTTP,
może połączyć się z serwerem, z dowolnego urządzenia mającego dostęp do internetu oraz do przeglądarki
internetowej, w celu wysłania zdjęcia na serwer.
Jeżeli to urządzenie posiada również dostęp do kamery, to jest
możliwość bezpośredniego wysłania zrobionego zdjęcia od razu do systemu.
W odpowiedzi na wysłane zdjęcie zostanie zwrócony komunikat, zawierający kim jest osoba przedstawiona na zdjęciu,
pod warunkiem, że ta osoba została wcześniej wgrana do systemu.

Dużym ułatwieniem w zarządzaniu bazą zdjęć jest przygotowany panel administratora,
który pozwala na łatwe zarządzanie zdjęciami, które znajdują się w systemie.
Podgląd aktualnych lub dodawanie kolejnych zdjęć ogranicza się tylko do paru kliknięć.
Niestety ręczne wysyłanie bardzo dużej liczby zdjęć za pomocą panelu może wymagać sporo czasu,
dlatego w takim przypadku lepiej jest przygotować skrypt, który zaimportuje bezpośrednio zdjęcia z dysku do systemu.

Próg, po którego przekroczeniu, walidator oceniający podobieństwo uzna,
że dane zdjęcia nie są do siebie podobne został wyliczony na \num{10.2}.
Przy tej wartości, na badanym zbiorze, została osiągnięta maksymalna skuteczność wynosząca \num{92.3}\%.
W zależności od tego, gdzie napisany system zostałby użyty, wartość tą można zmodyfikować.
W przypadku, gdy system zostałby użyty w miejscu, gdzie błędy fałszywie pozytywne są niedopuszczalne,
wartość ta powinna zostać zmniejszona.
Spowoduje to, że ogólna skuteczność walidatora spadnie, natomiast częstotliwość występowania wyników fałszywie
pozytywnych zostanie \textbf{najprawdopodobniej} ograniczona.

Wymagania, jakie zostały podstawione systemowi w rozdziale pierwszym, zostały w pełni zrealizowane.
Wyniki testów pokazały, że aplikacja poprawnie rozpoznaje zdjęcia niezawierające twarzy,
jest w stanie również ocenić z wysokim prawdopodobieństwem czy osoba rozpoznana na zdjęciu jest znana systemowi
oraz zwrócić informację na temat osoby przesłanej do systemu po jej rozpoznaiu