\chapter*{Podsumowanie}
\addcontentsline{toc}{chapter}{Podsumowanie}

Architektura FaceNet znakomicie sprawdza się do zadań związanych z rozpoznawaniem twarzy.

Wektor cech, które jest generowane przez sieć FaceNet

Policzona maksymalna odległość pomiędzy zdjęciami, która może wystąpić,
aby porównywane zdjęcia twarzy były uznane za podobne to \num{10.2}.
Przy takiej długości została osiągnięta maksymalna trafność samego walidatora.
Pomyłka walidatora będzie oznaczać, że osoba, która w rzeczywistości nie znajduje się w obecnej bazie zdjęć,
zostanie zakwalifikowana do następnego etapu i zostanie dokonana kwalifikacja (wynik fałszywie pozytywny).
W przypadku, gdy system zostałby użyty w miejscu, gdzie taki błąd jest niedopuszczalny, należy zmniejszyć tę odległość.
Spowoduje to, że ogólna trafność walidatora spadnie, natomiast częstotliwość występowania wyników fałszywie
pozytywnych zostanie \textbf{najprawdopodobniej} ograniczona.

Przygotowany panel administratora jest bardzo dużym ułatwieniem,
które pozwala na łatwe zarządzanie zdjęciami, które znajdują się w systemie.
Podgląd aktualnych lub dodawania kolejnych zdjęć ogranicza się tylko do paru kliknięć.
Niestety manualne wysyłanie bardzo dużej liczby zdjęć za pomocą panelu będzie wymagać dużo czasu,
dlatego w takim przypadku lepiej jest przygotować skrypt, który zaimportuje bezpośrednio zdjęcia z dysku do systemu.

Udostępniona strona internetowa pozwala na komunikowanie się z serwerem
z każdego urządzenia z internetem z zainstalowaną przeglądarką internetową.
Jeżeli urządzenia posiada aparat to jest nawet możliwość bezpośredniego wysłania zdjęcia zaraz po jego wykonaniu.
Takie rozwiązanie daje dużą swobodę w korzystaniu.


Kolejnym tematem, który nie został poruszony, jest wydajność systemu przy bardzo dużej liczbie zdjęć.
Podczas testów w systemie wgranych było 5000 zdjęć i przy takiej liczbie system odpowiadał szybko (poniżej 1 sekundy),
natomiast nie zostało sprawdzone, co się stanie, gdy liczba ta wzrośnie 10 lub 100-krotnie.

