\section*{Streszczenie}

W rozdziale pierwszym zostały poruszone zagadnienia związane z rozpoznawaniem twarzy oraz został
postawiony cel napisania aplikacji, za pomocą której użytkownik będzie mógł uzyskać informację na temat
osoby przedstawionej na wysłanym zdjęciu pod warunkiem,
że do aplikacji zostały wcześniej dostarczone zdjęcia reprezentujące daną osobę.
W rozdziale drugim została omówiona koncepcja działania aplikacji,
a w rozdziale trzecim zostały wybrane i opisane
technologie, które zostaną użyte do jej napisania.
Rozdział czwarty szczegółowo opisuje proces implementacji poszczególnych elementów
aplikacji oraz omawia
sposób znalezienia progu, za pomocą której zostanie uznane,
że dana osoba jest znana/nieznana aplikacji.
Ostatni rozdział przedstawia sposób testowania
oraz zawiera analizę otrzymanych wyników.


\section*{Abstract}

In chapter one is discussed issues related to face recognition and is set the goal of writing an application
to help find information who is pictured on the sent photo if the person has been previously uploaded to the serwer.
The second chapter discusses the concept of the system operation
and in the third chapter is chosen and described technologies that will be used to meet the expectations.
The fourth chapter describes in detail the process of implementing individual
system components and discusses the methodology of finding the threshold,
by means of which the system recognizes that a given person is known.
The last chapter shows tests of implemented system and analysis the results.